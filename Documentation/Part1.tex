\section{Системный объект phased.MVDREstimator2D}

Пакет: phased

Двумерная MVDR (Capon) оценка пространственного спектра

\subsection{Описание}

Объект \textbf{MVDREstimator2D} вычилсяет двумерную оценку пространственного спектра сигнала с минимальной дисперсией без искажений (Minimum Variance Distortionless Response, MVDR). Эта DOA оценка называется также оценка Капона(Capon).

Для оценки пространственного спектра:

\begin{enumerate}
	\item[\sffamily a)] Определите и настройте двумерную оценку пространственного спектра MVDR. См. п. \ref{Construction}«Конструкция»
	\item[\sffamily б)] Вызовите \underline{step} для оценки пространственного спектра в соответствии со свойствами phased.MVDREstimator2D. Поведение \underline{step} специфично для каждого объекта в панели инструментов.
\end{enumerate}

\subsection{Конструкция}\label{Construction}

\underline{H = phased.MVDREstimator2D} создает двумерную MVDR оценку пространственного спектра Системный объект, H. Этот объект оценивает пространственный спектр сигнала, используя узкополосный MVDR формирователя луча. 

\underline{H = phased.MVDREstimator2D(Name,Value)} создает объект, H, в котором каждое указанное свойство \underline{Name}, установлено в указанное значение \underline{Value}. Вы можете указать дополнительные аргументы пары имя-значение (Name-Value) в любом порядке как \underline{Name1,Value1,...,NameN,ValueN}.

\subsection{Свойства}

\begin{longtable}{|p{6cm}|p{11cm}|} \hline
	SensorArray
	&
	Управление матрицей датчиков.
	\newline
	\newline
	Устанавливает матрицу датчиков как дескриптор. Массив датчиков в пакете phased должен быть массивом объектов. Массив не может содержать подмассивы.
	\newline
	\newline
	\textbf{По умолчанию:}\href{https://www.mathworks.com/help/phased/ref/phased.ula-class.html}%
	{phased.ULA} со значениями по умолчанию
	
	\noindent \\
	\hline
	PropagationSpeed
	&
	Скорость распространения сигнала
	\newline
	\newline
	Устанавливает скорость распространения сигнала в метрах в секунду как положительное скалярное значение.
	\newline
	\newline
	\textbf{По умолчанию:} Скорость света \\
	\hline
	OperatingFrequency
	&
	Рабочая частота системы
	\newline
	\newline
	Устанавливает рабочую частоту системы в герцах как положительное скалярное значение.
	\newline
	\newline
	\textbf{По умолчанию:} 3e8 \\
	\hline
	NumPhaseShifterBits
	&
	Число бит квантования фазовращателя
	\newline
	\newline
	Число бит используемые для квантования составляющей фазового сдвига в формирователе луча или веса вектора управления. Устанавливает число бит как неотрицательное целое число. Нуль указывает, что квантование не выполняется.
	\newline
	\newline
	\textbf{По умолчанию:} 0 \\
	\hline
	ForwardBackwardAveraging
	&
	Выполнять ли усреднение прямого и обратного хода
	\newline
	\newline
	Установите для этого свойства значение \underline{true}, чтобы использовать усреднение прямого и обратного хода для оценки ковариационной матрицы для массивов датчиков с симметрично сопряженным решеточным множеством (array manifold).
	\newline
	\newline
	\textbf{По умолчанию:} false \\
	\hline
	AzimuthScanAngles
	&
	Углы сканирования по азимуту
	\newline
	\newline
	Устанавливает углы сканирования по азимуту (в градусах) как вектор действительных чисел. Углы должны быть указаны в интервале от -180 до 180 включительно. Указывать углы необходимо в порядке возрастания.
	\newline
	\newline
	\textbf{По умолчанию:} -90:90 \\
	\hline
	ElevationScanAngles
	&
	Углы сканирования по углу места
	\newline
	\newline
	Устанавливает углы сканирования по углу места (в градусах) как вектор действительных чисел или скаляр. Углы должны быть указаны в интервале от -90 до 90 включительно. Указывать углы необходимо в порядке возрастания.
	\newline
	\newline
	\textbf{По умолчанию:} 0 \\
	\hline
	DOAOutputPort
	&
	Разрешить вывод DOA
	\newline
	\newline
	Чтобы получить сигнал определения направления на источник (DOA, Destination of Arrival), установите для этого свойства значение \underline{true} и используйте соответствующий выходной аргумент при вызове \underline{step}. Если вы не хотите получать DOA, установите для этого свойства значение \underline{false}.
	\newline
	\newline
	\textbf{По умолчанию:} false \\
	\hline
	
	NumSignals
	&
	Количество сигналов
	\newline
	\newline
	Устанавливает количество сигналов для оценки DOA как положительное целое число. Это свойство применяется, если для свойства \underline{DOAOutputPort} установлено значение \underline{true}.
	\newline
	\newline
	\textbf{По умолчанию:} 1 \\
	\hline
\end{longtable}

\subsection{Методы}

\begin{longtable}{|p{6cm}|p{11cm}|} \hline
	plotSpectrum
	&
	График пространственного спектра \\ 
	\hline
	reset
	&
	Сброс состояния объекта двумерной MVDR оценки пространственного спектра \\ 
	\hline
	step
	&
	Выполнение оценки пространственного спектра \\ 
	\hline
\end{longtable}

\subsubsection{step}

Выполнение оценки пространственного спектра

\textbf{Синтаксис:}
\begin{center}
Y = step(H, X)

[Y, ANG] = step(H, X)
\end{center}

\textbf{Описание:}

Y = step(H, X) оценивает пространственный спектр из X с использованием оценки H. Где X - матрица, столбцы которой соответствуют каналам. Y - матрица, представляющая величину оцененного двумерного пространственного спектра. Колличество строк Y равно числу углов в \underline{ElevationScanAngles}, а колличество столбцов Y равно числу углов в свойстве \underline{AzimuthScanAngles}.

Размер первого измерения этой входной матрицы может изменяться для имитации изменяющейся длины сигнала, например импульсный сигнал с переменной частотой повторения импульсов.

[Y, ANG] = step(H, X) возвращает дополнительный выходной параметр \underline{ANG} в качестве направления поступления сигнала (DOA), когда свойство DOAOutputPort истинно. ANG - это двумерная матрица, где первая строка представляет оцененный азимут, а вторая - оцененный угол места (в градусах).